\chapter{Глава 1}
\section{Индустрия 4.0}
Четвертая промышленная революция


\section{Обзор решений}
Пара слов
\subsection{OpenSource}
FrePPLe, Odoo, qcadoo?
По презентации
\subsection{Commercial}
Нужно найти
\section{Постановка задач}
\todo[inline, color=yellow]{В целом постановка задачи}
\subsection{Календарь}
В соответствии с архитектурой, представленной ранее, расчет выполнения операции
(или набора операций \todo{Пояснение ТК}) производится в логическом времени,
т.~е. во времени отсчитываемому от нуля. Данное решение обуславливает необходимость в
отображении (\todo{Пояснение отображения}) логического времени на физическое, 
которое используется в повседневной жизни(\todo{Пояснение} ).
Одной из главных сложностей, возникающих при этом, является неоднородность рабочего 
времени, которая проявляется в (\todo{сменность графика по терминологии}), 
наличии выходных, перенесенных дней. Другой сложностью является наличие в системе 'обратного расчета', при котором планирование ведется от даты \todo{'дедлайна'}, что накладывает некоторые ограничения на реализацию данной компоненты.
\subsection{Ресурс сборочной линии}
\todo[inline]{Подумать над формулировкой}
Данная компонента является частью (или частной реализацией) модели ресурсов СПП и отражает поведение во времени продуктов на сборочной линии. Основная сложность данной компоненты в необходимости объединения нескольких сборочных линий со схожими параметрами в один ресурс, хранении их состояний, синхронизации между \todo{станциями} и распределении \todo{входного потока продуктов} по линиям. Также, сложностью является сама модель ресурса, которая обеспечивает хорошую масштабируемость, но при этом требует времени на понимание и создание модулей.
\subsection{База данных}
База данных является основных хранилищем всех постоянных \todo{(не расчетных, как, например, результат работы СПП), так ли это?} данных. При разработке было выделено требование к хранению \todo{'истории'} предыдущих расчетов и их параметров, что влечет к требованию \todo{'консистентности'} \todo{базы данных, БД?}
