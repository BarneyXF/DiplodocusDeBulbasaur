\chapter{GPU}
\label{cha:chapter}

%\todo[inline, color=red]{так выглядит заметка в тексте}
\section{Цены на облака}

\textbf{Microsoft Azure}

При оплате по мере использования: от 0,8 до 3,4 за час.
При резервировании на год: от 0,5 до 2,176 за час.
При резервировании на три года: от 0,352 до 1,5 за час.

\textbf{AWS EC2}

Оплата по факту использования за инстанс с GPU: от 0,75 до 14,4 за час.

\textbf{Amazon SageMaker}

От 0,134 до 6,4 за час

\textbf{Сirrascale}

От 3 до 12 за час.

\textbf{Google Cloud}

При оплате по факту: от 0,49 до 2,55 за час.
При предоплате: от 0,135 до 0,74 за час.

\textbf{Reg.ru}

От 2500 рублей в день (от 1.6 за час)

\textbf{mail.ru}

От 2.4 за час (минимальное использование - один год).

\section{Собрать машину самому}

\textbf{Игровой}

Процессор Intel Cоre i7 - 35000

Видеокарта GIGABYTE nVidia GeForce RTX 2080 - 70000

SSD 240 Гб - 11000

Оперативная память 2х16 Гб - 32000

Итого: 148 000 = 2242 USD.

При цене облака в 3 USD/час окупится за 747 часов при сопоставимой производительности.

При пользовании 20 ч в месяц - окупится примерно за 3 года.

\textbf{Самая дорогая комплектуха в Citilink}

Процессор Intel Cоre i9 9960X - 137000

Видеокарта  GeForce RTX 2080 Ti - 108000

SSD 8000 Гб - 175000

Оперативная память 8х16 Гб - 144000

Итого: 564 000 = 8545 USD

При цене облака в 3 USD/час окупится за 2848 часов при сопоставимой производительности.

При пользовании 20 ч в месяц - окупится примерно за 11 лет.

\textbf{Готовая сборка} 

Компьютер ACER Predator PO9-900 - 750 000 = 11360 USD

При пользовании 20 ч в месяц - окупится примерно за 15 лет.

\textbf{Оборудование, используемое в облачных}

Процессор INTEL Xeon Processor E5-2660 v4 OEM - 105 000 = 1600 USD

Модуль вычислительный PNY Tesla P100 - 378 000 = 5700 USD


\textbf{Жиленков}

nVidia DGX Station 50 000 - 70 000 USD

сервер из DGX1 (на Pascal - около 130 000 или Volta - 150 000)  

DGX-2 - около 400 000
