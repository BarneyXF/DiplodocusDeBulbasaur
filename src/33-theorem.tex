\section{Формализация свойств базы данных}\label{sec:theorem}
\indent Пусть заданы два отношения $R_1$ и $R_2$, такие что:
\begin{eqnarray*}
	R_1 & = & <name, version, ...> \\
	R_2 & = & <FK(R_1, name), version, ...>
\end{eqnarray*}
\indent $FK(R, a)$ -- внешний ключ, ссылающийся на атрибут $a$ на отношения $R$.\\
\indent Также для атрибута $version$ данных отношений задано условие монотонного возрастания:
\begin{equation}
	\forall t_1 \in \mathbb{Z}, t_2 \in \mathbb{Z}, f: M \in \mathbb{R} \rightarrow \mathbb{R} : t_2 > t_1 \Rightarrow f(t_2) > f(t_1)
\end{equation}

\begin{definition}
	\label{def:join}
	Версионное соединение - это такое соединение двух отношений $R_1$ и $R_2$, для которых справедливо:\todo{условие максимальности наверное не нужно}
	\begin{multline}
		\label{eq:versJoin}
		\forall R_1,\ R_2\ :\ R_1.name\ \supset\ R_2.name:\\
		R_1\ \overrightarrow{\bowtie}\ R_2\ \Leftrightarrow\ \pi_{*, F_{MAX(R_1.version)}}(\sigma_{R_1.version \leq R_2.version}(R_1\ \bowtie_{R_1.name = R_2.name}\ R_2))
		% R_1,\ R_2 \Rightarrow R_1\ \overrightarrow{\bowtie}\ R_2\
	\end{multline}
\end{definition}

\todo{пофиксить выравнивание}
\begin{definition}
	\label{def:union}
	Пересечение двух версионных соединений $R_1$ и $R_2$ осуществляется по следующему правилу:
	\begin{equation*}
		\overrightarrow{\cup} ::\ R\overrightarrow{\bowtie}R' \rightarrow R\overrightarrow{\bowtie}R' \rightarrow R\overrightarrow{\bowtie}R'
	\end{equation*}
	\begin{multline}
		\forall R_1,R_2,R_1',R_2': R_2'.name \subset (R_1.name \cup R_1'.name),\\
		R_1.version \leq R_2'.version, R_1'.version > R_2.version:
	\end{multline}
	\begin{equation}
		(R_1\overrightarrow{\bowtie}\ R_2) \overrightarrow{\cup} (R_1'\ \overrightarrow{\bowtie}\ R_2') = (R_1 \cup R_1')\ \overrightarrow{\bowtie}\ (R_2 \cup R_2')
	\end{equation}
	% \begin{equation*}
		% 	\overrightarrow{\cap} ::\ R\overrightarrow{\bowtie}R' \rightarrow R\overrightarrow{\bowtie}R' \rightarrow R\overrightarrow{\bowtie}R'
	% \end{equation*}
	% \begin{multline*}
		% 	\forall R_1, R_2, R_1', R_2' : R_2'.name \subset (R_1.name \cup R_1'.name),\\
		% 	R_1.version \leq R_2'.version, R_1'.version > R_2.version:
	% \end{multline*}
\end{definition}

% \begin{definition}
% 	Выборка из версионного соединения 
% \end{definition}
\indent Обозначим версионное соединение отношений $R_1$ и $R_2$, как $J$:
\begin{eqnarray*}
	R_1 \overrightarrow{\bowtie} R_2 & \Leftrightarrow & J \\
	R_1' \overrightarrow{\bowtie} R_2' & \Leftrightarrow & J'
\end{eqnarray*}

\begin{theorem}
	\label{th:th1}
	Любая выборка из версионного соединения $J$ входит в выборку из объединения $J$ и любого допустимого версионного соединения $J'$ (отношения $R_1$ и $R_1'$ и $R_2$ и $R_2'$ одинаковы по набору атрибутов) по условиям $\overrightarrow{\cup}$.
	\begin{equation}
		\forall J, J': \sigma_\phi(J) \subset \sigma_\phi(J \overrightarrow{\cup} J')
	\end{equation}
\end{theorem}

\indent Если предположить, что теорема \ref{th:th1} неверна, тогда объединение $J$ и $J'$ не включают в себя $J$ для любого условия $\phi$.
Предположим что $\phi = true$, тогда выборка из $J$ будет содержать полное множество отношения $J$:
\begin{equation*}
	\label{eq:firstProof}
	\forall J: J \Leftrightarrow \sigma_{true}(J) \Rightarrow J \not\subset\ J\ \overrightarrow{\cup}\ J'
\end{equation*}
\indent По определению $J \Leftrightarrow R_1\ \overrightarrow{\bowtie}\ R_2$:
\begin{equation}
	R_1\ \overrightarrow{\bowtie}\ R_2 \not\subset\ (R_1\ \overrightarrow{\bowtie}\ R_2)\ \overrightarrow{\cup}\ (R_1'\ \overrightarrow{\bowtie}\ R_2')
\end{equation}
\indent По определению \ref{def:union}:
\begin{equation}
	\label{eq:notsbset}
	R_1\ \overrightarrow{\bowtie}\ R_2 \not\subset (R_1\ \cup\ R_2)\ \overrightarrow{\bowtie}\ (R_1'\ \cup\ R_2')
\end{equation}
\indent Так как отношения могут быть в том числе и пустыми, приравняем $R_1'$ и $R_2'$ пустому отношению, тогда выражение \ref{eq:notsbset} примет вид:
\begin{equation}
	\label{eq:proof}
	\left.
		\begin{array}{ccc}
			&R_1'\ & =\ \varnothing \\
			&R_2'\ & =\ \varnothing \\
		\end{array}
	\right\} \Rightarrow
	R_1\ \overrightarrow{\bowtie}\ R_2 \not\subset R_1\ \overrightarrow{\bowtie}\ R_2
\end{equation}
\indent Из выражения \ref{eq:proof} видно, что в результате получается, что версионное соединение отношений не входит само в себя, что неверно.
Из этого следует что изначальное предположение оказалось неверным, а следовательно теорема \ref{th:th1} верна, что и требовалось доказать.

\indent Пусть заданы три отношения $R_1$, $R_2$ и $R_3$, такие что:
\begin{eqnarray*}
	R_1 & = & <name, version, ...> \\
	R_2 & = & <FK(R_1, name), version, ...>\\
	R_3 & = & <FK(R_2, name), version, ...>
\end{eqnarray*}
\begin{theorem}
	\label{th:assoc}
	Версионное соединение лево ассоциативно.
	\begin{equation}
		R_1 \overrightarrow{\bowtie} (R_2 \overrightarrow{\bowtie} R_3) \Rightarrow R_1 \overrightarrow{\bowtie} R_3
	\end{equation}
\end{theorem}
\todo{посмотреть по терминологии подходит ли название ассоциативности}


\indent Так как известно, что $R_2.name \subset R_1.name$, а $R_3.name \subset R_2.name$, тогда $R_3.name \subset R_1.name$.
Также известно из определения \ref{def:join}, из условия выборки, что $R_1.version \leq R_2.version$ и $R_2.version \leq R_3.version$, что означает корректность выражения $R_1.version \leq R_3.version$.
В итоге получается, что совпадают оба условия существования версионного соединения для $R_1$ и $R_2$, подтверждает теорему \ref{th:assoc}




%%%%%%%%%%%%%%%%%%%%%%%%%%%%%%%%%%%%%%%%%%%%%%%%%%%%%%%%%%%%%%%%%%%%%%%%%%%%%%%
% \indent Формула \ref{eq:versJoin} отражает эквивалентность ``версионного соединения'' и связанных по условию отношений $R_1$ и $R_2$.
% То есть для версионного соединения должны выполняться два условия:
% \begin{itemize}
% 	\item внешний ключ $R_2.name$ второго отношения существует в домене $R_1.name$;
% 	\item версия указанная во втором отношении не меньше версии указанной в первом.
% \end{itemize}

% \indent При их соблюдении возможно применять свойство следующее из определения версионного соединения:
% \begin{equation}
% 	\forall\ f : f(R_1\ \overrightarrow{\bowtie}\ R_2,\ R_1'\ \overrightarrow{\bowtie}\ R_2') \Leftrightarrow
% 	f(R_1,\ R_1')\ \overrightarrow{\bowtie}\ f(R_2,\ R_2')
% \end{equation}
% \indent Другое свойство версионного соединения: 
% \begin{multline}
% 	\label{eq:prop2}
% 	\forall R, R', R_1', R_2', f : R \varsubsetneq f(R, R'), R_1\ \overrightarrow{\bowtie}\ R_2, R_2'.name \subset R_1.name :\\
% 	\left.
% 	\begin{array}{ccc}
% 		f(R_1, R_1') \Rightarrow \bot \\
% 		f(R_2, R_2') \Rightarrow R_2'' \\
% 		f(R_1, R_1'), f(R_2, R_2') \Rightarrow R_1'', R_2''
% 	\end{array} 
% 	\right\} \Rightarrow
% 	\left.
% 	\begin{array}{ccc}
% 		\bot \\
% 		R_1\ \overrightarrow{\bowtie}\ R_2''\\
% 		R_1''\ \overrightarrow{\bowtie}\ R_2''
% 	\end{array}
% 	\right.
% \end{multline}
% \to\do{как это адекватно записать?}
% \indent В выражении \ref{eq:prop2} показано свойство, что изменение множества кортежей отношения $R_1$ функцией $f(R, R')$ недопустимо, при том что либо только $R_2$ либо и то и другое отношение одновременно могут быть изменены при соблюдении условий версионного соединения.\\
% \indent Оператор версионного соединения ($\overrightarrow{\bowtie}$) не следует понимать как обычное пересечение ($\bowtie$), потому как он лишь отражает взаимосвязь двух отношений не соединяя их в одно.

% \begin{theorem}
% 	Любая выборка ($\sigma_\phi$) из отношения полученного путем объединения двух версионных соединений $R$ и $R'$ будет включать в себя выборку по тому же условию из $R$.
% \end{theorem}
% \begin{eqnarray}
% 	&R\ & \leftarrow (R_1\ \overrightarrow{\bowtie}\ R_2),\ 
% 	R'\ \leftarrow (R'_1\ \overrightarrow{\bowtie}\ R'_2)\\
% 	&f:: & R_1 \overrightarrow{\bowtie}\ R_2 \rightarrow R_1\ \overrightarrow{\bowtie}\ R_2 \rightarrow R_1\ \overrightarrow{\bowtie}\ R_2 \\
% 	&\forall\ & R,\ R', f : R\ \subsetneq\ f (R,\ R') \\
% 	&\forall\ & \phi,\ R,\ R' :\ \sigma_\phi(R) \subset \sigma_\phi(f(R,\ R'))
% 	\label{eq:th}
% \end{eqnarray}

% \indent Если предположить, что теорема \ref{eq:th} неверна, тогда объединение $R$ и $R'$ не включают в себя $R$ для любого условия $\phi$:
% \t\odo{декомпозировать}
% \begin{eqnarray}
% 	&R& \not\subset\ f(R,\ R') \\
% 	&R& \not\subset\ R\ \cup\ R' \\
% 	&R& \not\subset\ (R_1\ \overrightarrow{\bowtie}\ R_2)\ \cup\ (R_1'\ \overrightarrow{\bowtie}\ R_2') \\
% 	&R_1& \overrightarrow{\bowtie}\ R_2 \not\subset (R_1\ \cup\ R_2)\ \overrightarrow{\bowtie}\ (R_1'\ \cup\ R_2')\label{eq:wrong} \\
% \end{eqnarray}
% \indent Так как множества могут быть в том числе и пустыми, приравняем $R_1'$ и $R_2'$ пустому множеству:
% \begin{equation}
% 	\label{eq:proof}
% 	\left.
% 		\begin{array}{ccc}
% 			&R_1'\ & =\ \varnothing \\
% 			&R_2'\ & =\ \varnothing \\
% 		\end{array}
% 	\right\} \Rightarrow
% 	R_1\ \overrightarrow{\bowtie}\ R_2 \not\subset R_1\ \overrightarrow{\bowtie}\ R_2
% \end{equation}
% \indent Из выражения \ref{eq:proof} видно, что в результате получается, что версионное соединение множеств не входит само в себя, что неверно.
% Из этого следует что изначальное предположение оказалось неверным, а следовательно теорема \ref{eq:th} доказана.

% \begin{equation*}
	% 	\forall name \in R_1.name : \forall v_1 \in R_1.version v_2 \in R_1.version: v_1 \neq v_2 :
% \end{equation*}
% \begin{equation}
		
% \end{equation}

\todo[inline]{Доказать}
% \to\do[inline]{монотонность версий}
\todo[inline]{оформить свойства и определение}