\section{Формализация свойств базы данных}\label{sec:theorem}
\indent Для достижения согласованности данных необходимо чтобы были запрещены операции модификации и удаления данных, а запись новых кортежей в базу данных обладала двумя свойствами:
\begin{enumerate}
	\item[1)] запись во все отношения позволяет получить любую предыдущую версию данных;
	\item[2)] корректная запись в одно из отношений (при соблюдении соответствующих условий, о которых речь пойдет далее) не нарушает согласованности в базе данных.
\end{enumerate}

\indent Пусть заданы два отношения $R_1$ и $R_2$, такие что:
\begin{eqnarray*}
	R_1 & = & <name_1, version, ...> \\
	R_2 & = & <FK(R_1, name_1), version, ...>
\end{eqnarray*}

\indent Для атрибута $version$ данных отношений задано требование монотонного возрастания:
\begin{equation}
	\forall t_1 \in \mathbb{Z}, t_2 \in \mathbb{Z}, f: M \in \mathbb{Z} \rightarrow \mathbb{Z} : t_2 > t_1 \Rightarrow f(t_2) > f(t_1)
\end{equation}

\indent Например, в качестве самой простой и очевидной монотонно возрастающей функции можно взять функцию линей зависимости версии от времени $version(t) = t$, где $t$ будет количеством секунд с начала эпохи Unix.
% \todo{необходима корректировка}


\begin{definition}
	\label{def:join}
	Версионное соединение -- это такое соединение двух отношений $R_1$ и $R_2$, для которых справедливо:%\todo{условие максимальности наверное не нужно}
	\begin{multline}
		\label{eq:versJoin}
		\forall R_1,\ R_2\ :\ R_1.name_1\ \supset\ R_2.name_1:\\
		R_1\ \overrightarrow{\bowtie}\ R_2\ \Leftrightarrow\ _{R_2.version}\gamma(\sigma_{R_1.version \leq R_2.version}(R_1\ \bowtie_{R_1.name_1 = R_2.name_1}\ R_2))
		% R_1,\ R_2 \Rightarrow R_1\ \overrightarrow{\bowtie}\ R_2\
	\end{multline}
\end{definition}

\indent Также будем обозначать отношение $\forall n : R_n'$ как отношение которое полностью повторяющее набор атрибутов $R_n$.

\begin{definition}
	\label{def:union}
	Объединение двух версионных соединений $R_1$ и $R_2$ осуществляется по следующему правилу:
	\begin{equation*}
		\overrightarrow{\cup} :\ R_1\overrightarrow{\bowtie}R_2' \rightarrow R_1\overrightarrow{\bowtie}R_2' \rightarrow R_1\overrightarrow{\bowtie}R_2'
	\end{equation*}
	\begin{multline*}
		\forall R_1,R_2,R_1',R_2': R_2'.name \subset (R_1.name \cup R_1'.name),\\
		R_1.version \leq R_2'.version, R_1'.version > R_2.version:
	\end{multline*}
	\begin{equation}
		(R_1\overrightarrow{\bowtie}\ R_2) \overrightarrow{\cup} (R_1'\ \overrightarrow{\bowtie}\ R_2') = (R_1 \cup R_1')\ \overrightarrow{\bowtie}\ (R_2 \cup R_2')
	\end{equation}
	Во всех случаях не описанных выше объединение не может быть произведено.
	% \begin{equation*}
		% 	\overrightarrow{\cap} ::\ R\overrightarrow{\bowtie}R' \rightarrow R\overrightarrow{\bowtie}R' \rightarrow R\overrightarrow{\bowtie}R'
	% \end{equation*}
	% \begin{multline*}
		% 	\forall R_1, R_2, R_1', R_2' : R_2'.name \subset (R_1.name \cup R_1'.name),\\
		% 	R_1.version \leq R_2'.version, R_1'.version > R_2.version:
	% \end{multline*}
\end{definition}

\indent Обозначим версионное соединение отношений $R_1$ и $R_2$, как $\overrightarrow{J}$:
\begin{eqnarray*}
	R_1 \overrightarrow{\bowtie} R_2 & \Leftrightarrow & \overrightarrow{J} \\
	R_1' \overrightarrow{\bowtie} R_2' & \Leftrightarrow & \overrightarrow{J}'
\end{eqnarray*}

\begin{theorem}
	\label{th:th1}
	Любая выборка из версионного соединения $\overrightarrow{J}$ входит в выборку из объединения $\overrightarrow{J}$ и любого допустимого версионного соединения $\overrightarrow{J}'$ (отношения $R_1$ и $R_1'$ и $R_2$ и $R_2'$ одинаковы по набору атрибутов) по условиям $\overrightarrow{\cup}$.
	\begin{equation}
		\forall \overrightarrow{J}, \overrightarrow{J}': \sigma_\phi(\overrightarrow{J}) \subset \sigma_\phi(\overrightarrow{J} \overrightarrow{\cup} \overrightarrow{J}')
	\end{equation}
\end{theorem}
\begin{proof}
	\indent Если предположить, что теорема \ref{th:th1} неверна, тогда объединение $\overrightarrow{J}$ и $\overrightarrow{J}'$ не включают в себя $\overrightarrow{J}$ для любого условия $\phi$.
	Предположим что $\phi = true$, тогда выборка из $\overrightarrow{J}$ будет содержать полное множество отношения $\overrightarrow{J}$:
	\begin{equation*}
		\label{eq:firstProof}
		\forall \overrightarrow{J}: \overrightarrow{J} \Leftrightarrow \sigma_{true}(\overrightarrow{J}) \Rightarrow \overrightarrow{J} \not\subset\ \overrightarrow{J}\ \overrightarrow{\cup}\ \overrightarrow{J}'
	\end{equation*}
	\indent По определению $\overrightarrow{J} \Leftrightarrow R_1\ \overrightarrow{\bowtie}\ R_2$:
	\begin{equation}
		R_1\ \overrightarrow{\bowtie}\ R_2 \not\subset\ (R_1\ \overrightarrow{\bowtie}\ R_2)\ \overrightarrow{\cup}\ (R_1'\ \overrightarrow{\bowtie}\ R_2')
	\end{equation}
	\indent По определению \ref{def:union}:
	\begin{equation}
		\label{eq:notsbset}
		R_1\ \overrightarrow{\bowtie}\ R_2 \not\subset (R_1\ \cup\ R_1')\ \overrightarrow{\bowtie}\ (R_2\ \cup\ R_2')
	\end{equation}
	
	\indent По определению объединения множеств $R_1 \cup R_2 = \{x~|~x~\in
	~R_1\ |\ x~\in~R_2\}$, что говорит о том, что результирующее множество содержит полное множество $R_1$ и $R_2$, следовательно :
	
	\begin{equation}
		\label{eq:proof}
		\forall R_1', R_2':
		\left.
		\begin{array}{ccc}
			&R_1\ & \not\subset\ (R_1 \cup R_1') \\
			&R_2\ & \not\subset\ (R_2 \cup R_2') \\
		\end{array}
		\right.
		% \left.
		% 	\begin{array}{ccc}
			% 		&R_1'\ & =\ \varnothing \\
			% 		&R_2'\ & =\ \varnothing \\
		% 	\end{array}
		% \right\} \Rightarrow
		% R_1\ \overrightarrow{\bowtie}\ R_2 \not\subset R_1\ \overrightarrow{\bowtie}\ R_2
	\end{equation}
	\indent Выражения \ref{eq:proof} опровергают определение объединения множеств, что приводит к выводу о неверности предположения.
	Из этого следует, что теорема \ref{th:th1} верна, что и требовалось доказать.
\end{proof}

\indent Пусть заданы три отношения $R_1$, $R_2$ и $R_3$, такие что:
%\todo{Дописать}
\begin{eqnarray*}
	R_1 & = & <name_1, version_1, ...> \\
	R_2 & = & <FK(R_1, name_1), name_2, version_2, ...>\\
	R_3 & = & <FK(R_2, name_2), version_3, ...>
\end{eqnarray*}
\begin{theorem}
	\label{th:assoc}
	Для любого версионного объединения $(R_2 \overrightarrow{\bowtie} R_3 ) \overrightarrow{\cup} (R_2' \overrightarrow{\bowtie} R_3')$ существует корректное версионное соединение получившегося отношения с $R_1$.
	\begin{equation}
		R_1 \overrightarrow{\not\bowtie} (R_2 \overrightarrow{\bowtie} R_3) \Rightarrow R_1 \overrightarrow{\not\bowtie} ((R_2 \overrightarrow{\bowtie} R_3 ) \overrightarrow{\cup} (R_2' \overrightarrow{\bowtie} R_3'))
	\end{equation}
\end{theorem}
\begin{proof}
	\indent Пусть не существует верное соединение при верном версионном объединении:
	\begin{equation}
		\label{eq:notbowtie}
		R_1 \overrightarrow{\not\bowtie} (R_2 \overrightarrow{\bowtie} R_3 ) \overrightarrow{\cup} (R_2' \overrightarrow{\bowtie} R_3'))
	\end{equation}
	
	\indent Следовательно не выполняется одно либо оба условия существования версионного соединения:
	\begin{enumerate}
		\item[1)] $((R_2 \overrightarrow{\bowtie} R_3 ) \overrightarrow{\cup} (R_2' \overrightarrow{\bowtie} R_3')).name_1 \not\subset R_1.name_1$;
		\item[2)] $((R_2 \overrightarrow{\bowtie} R_3 ) \overrightarrow{\cup} (R_2' \overrightarrow{\bowtie} R_3')).version_3 < R_1.version_1$
	\end{enumerate}
	\indent Если рассматривать первый вариант, то необходимо, используя определение \ref{def:union}, упростить выражение \ref{eq:notbowtie}:
	\begin{equation}
		R_1 \overrightarrow{\not\bowtie} ((R_2 \cup R_2') \overrightarrow{\bowtie} (R_3 \cup R_3'))
	\end{equation}
	\indent Тогда, используя определение отношений:
	\begin{equation}
		(R_2 \cup R_2').name_1 \not\subset R_1.name_1
	\end{equation}
	\indent Так как в начальный момент времени система находилась в состоянии согласованности, то $R_2.name_1 \subset R_1.name_1 \Rightarrow R_2' \not\subset R_1.name_1$.
	Данное утверждение является нарушением целостности по ссылкам, так что любая система управления базами данных запретит объединение $R_2 \cup R_2'$.\\
	\indent Рассматривая второе условие:
	\begin{equation}
		((R_2 \cup R_2') \overrightarrow{\bowtie} (R_3 \cup R_3')).version_3 < R_1.version_1
	\end{equation}
	\indent Тогда, используя определение отношений:
	\begin{equation}
		(R_3 \cup R_3').version_3 < R_1.version_1
	\end{equation}
	\indent Так как в начальный момент времени система находилась в состоянии согласованности, то:
	\begin{equation}
		R_1.version_1 \leq R_3.version3 \Rightarrow
		\begin{cases}
			R_3'.version_3 < R_1.version_1 \\
			R_3'.version_3 < R_3.version_3
		\end{cases}
	\end{equation} 
	\indent Из данной системы следует нарушение монотонности возрастания значения версий (новая версия меньше старой), что является нарушением условий версионного объединения.\\
	\indent В результате, оба варианта получения несогласованного состояния в базе данных ограничиваются либо системой управления базами данных, либо заданными определениями, что доказывает теорему \ref{th:assoc}.
	\indent Так как теорема \ref{th:assoc} справедлива для 3 отношений, можно сказать что данная теорема будет справедлива и для $\forall N \in \mathbb{Z}$.
\end{proof}

\indent Пусть есть $N$ отношений таких, что:
\begin{eqnarray*}
	R_1 & = & <name_1, version_1, ...> \\
	R_2 & = & <FK(R_1, name_1), name_2, version_2, ...>\\
	R_3 & = & <FK(R_2, name_2), name_3, version_3, ...>\\
	&...&\\
	R_N & = & <FK(R_{N-1}, name_{N-1}), version_N, ...>
\end{eqnarray*}
\indent Тогда, из теоремы \ref{th:assoc} следует, что первое условие существования версионного объединения никак не меняется и полностью зависит от $R_2$, которое в свою очередь каскадно зависит от последующих отношений.
Второе условие, по причине того, что в результирующем версионном объединении в качестве версии будет выбрано самое последнее значение, то есть $R_N.version_N$, соответственно если взять $N = 3$, то получится уже доказанная теорема, из чего следует, что теорема справедлива для $\forall N \in \mathbb{Z}$.
% \todo{индукция}


%%%%%%%%%%%%%%%%%%%%%%%%%%%%%%%%%%%%%%%%%%%%%%%%%%%%%%%%%%%%%%%%%%%%%%%%%%%%%%%
% \indent Здесь намеренно упущены пункты про чтение данных и что они должны соответствовать названным свойствам, потому как подразумевается, что когда не происходит никаких операций способных изменить данные (к которым относится чтение), то они остаются в согласованном состоянии.
% \indent Так как отношения могут быть в том числе и пустыми, приравняем $R_1'$ и $R_2'$ пустому отношению, тогда выражение \ref{eq:notsbset} примет вид:
% \todo{посмотреть по терминологии подходит ли название ассоциативности}



% \indent Так как известно, что $R_2.name \subset R_1.name$, а $R_3.name \subset R_2.name$, тогда $R_3.name \subset R_1.name$.
% Также известно из определения \ref{def:join}, из условия выборки, что $R_1.version \leq R_2.version$ и $R_2.version \leq R_3.version$, что означает корректность выражения $R_1.version \leq R_3.version$.
% В итоге получается, что совпадают оба условия существования версионного соединения для $R_1$ и $R_3$, подтверждает теорему \ref{th:assoc}
% \indent Из выражения \ref{eq:proof} видно, что в результате получается, что версионное соединение отношений не входит само в себя, что неверно.
% Из этого следует что изначальное предположение оказалось неверным, а следовательно теорема \ref{th:th1} верна, что и требовалось доказать.
% \indent Как было доказано, после применения операции версионного объединения есть возможность получить предыдущее состояние отношений, что отвечает требованиям консистентности, а так как получается, что единственная операция, которая может менять отношения консистентна, то можно говорить что и полученная база данных консистентна.
% \indent Исходя из требований консистентности к разрабатываемой базе данных, то есть о двух и только двух определенных операциях: чтения и записи, и из того, что чтение не изменяет данных, а запись, как было доказано позволяет получить 
% \indent Формула \ref{eq:versJoin} отражает эквивалентность ``версионного соединения'' и связанных по условию отношений $R_1$ и $R_2$.
% То есть для версионного соединения должны выполняться два условия:
% \begin{itemize}
	% 	\item внешний ключ $R_2.name$ второго отношения существует в домене $R_1.name$;
	% 	\item версия указанная во втором отношении не меньше версии указанной в первом.
% \end{itemize}

% \indent При их соблюдении возможно применять свойство следующее из определения версионного соединения:
% \begin{equation}
% 	\forall\ f : f(R_1\ \overrightarrow{\bowtie}\ R_2,\ R_1'\ \overrightarrow{\bowtie}\ R_2') \Leftrightarrow
% 	f(R_1,\ R_1')\ \overrightarrow{\bowtie}\ f(R_2,\ R_2')
% \end{equation}
% \indent Другое свойство версионного соединения: 
% \begin{multline}
% 	\label{eq:prop2}
% 	\forall R, R', R_1', R_2', f : R \varsubsetneq f(R, R'), R_1\ \overrightarrow{\bowtie}\ R_2, R_2'.name \subset R_1.name :\\
% 	\left.
% 	\begin{array}{ccc}
% 		f(R_1, R_1') \Rightarrow \bot \\
% 		f(R_2, R_2') \Rightarrow R_2'' \\
% 		f(R_1, R_1'), f(R_2, R_2') \Rightarrow R_1'', R_2''
% 	\end{array} 
% 	\right\} \Rightarrow
% 	\left.
% 	\begin{array}{ccc}
% 		\bot \\
% 		R_1\ \overrightarrow{\bowtie}\ R_2''\\
% 		R_1''\ \overrightarrow{\bowtie}\ R_2''
% 	\end{array}
% 	\right.
% \end{multline}
% \to\do{как это адекватно записать?}
% \indent В выражении \ref{eq:prop2} показано свойство, что изменение множества кортежей отношения $R_1$ функцией $f(R, R')$ недопустимо, при том что либо только $R_2$ либо и то и другое отношение одновременно могут быть изменены при соблюдении условий версионного соединения.\\
% \indent Оператор версионного соединения ($\overrightarrow{\bowtie}$) не следует понимать как обычное пересечение ($\bowtie$), потому как он лишь отражает взаимосвязь двух отношений не соединяя их в одно.

% \begin{theorem}
% 	Любая выборка ($\sigma_\phi$) из отношения полученного путем объединения двух версионных соединений $R$ и $R'$ будет включать в себя выборку по тому же условию из $R$.
% \end{theorem}
% \begin{eqnarray}
% 	&R\ & \leftarrow (R_1\ \overrightarrow{\bowtie}\ R_2),\ 
% 	R'\ \leftarrow (R'_1\ \overrightarrow{\bowtie}\ R'_2)\\
% 	&f:: & R_1 \overrightarrow{\bowtie}\ R_2 \rightarrow R_1\ \overrightarrow{\bowtie}\ R_2 \rightarrow R_1\ \overrightarrow{\bowtie}\ R_2 \\
% 	&\forall\ & R,\ R', f : R\ \subsetneq\ f (R,\ R') \\
% 	&\forall\ & \phi,\ R,\ R' :\ \sigma_\phi(R) \subset \sigma_\phi(f(R,\ R'))
% 	\label{eq:th}
% \end{eqnarray}

% \indent Если предположить, что теорема \ref{eq:th} неверна, тогда объединение $R$ и $R'$ не включают в себя $R$ для любого условия $\phi$:
% \t\odo{декомпозировать}
% \begin{eqnarray}
% 	&R& \not\subset\ f(R,\ R') \\
% 	&R& \not\subset\ R\ \cup\ R' \\
% 	&R& \not\subset\ (R_1\ \overrightarrow{\bowtie}\ R_2)\ \cup\ (R_1'\ \overrightarrow{\bowtie}\ R_2') \\
% 	&R_1& \overrightarrow{\bowtie}\ R_2 \not\subset (R_1\ \cup\ R_2)\ \overrightarrow{\bowtie}\ (R_1'\ \cup\ R_2')\label{eq:wrong} \\
% \end{eqnarray}
% \indent Так как множества могут быть в том числе и пустыми, приравняем $R_1'$ и $R_2'$ пустому множеству:
% \begin{equation}
% 	\label{eq:proof}
% 	\left.
% 		\begin{array}{ccc}
% 			&R_1'\ & =\ \varnothing \\
% 			&R_2'\ & =\ \varnothing \\
% 		\end{array}
% 	\right\} \Rightarrow
% 	R_1\ \overrightarrow{\bowtie}\ R_2 \not\subset R_1\ \overrightarrow{\bowtie}\ R_2
% \end{equation}
% \indent Из выражения \ref{eq:proof} видно, что в результате получается, что версионное соединение множеств не входит само в себя, что неверно.
% Из этого следует что изначальное предположение оказалось неверным, а следовательно теорема \ref{eq:th} доказана.

% \begin{equation*}
	% 	\forall name \in R_1.name : \forall v_1 \in R_1.version v_2 \in R_1.version: v_1 \neq v_2 :
% \end{equation*}
% \begin{equation}
		
% \end{equation}

% \todo[inline]{Доказать}
% \to\do[inline]{монотонность версий}
% \todo[inline]{оформить свойства и определение}