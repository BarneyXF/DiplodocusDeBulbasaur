\renewcommand\appendixname{Приложение~В}
\chapter*{~~Модуль отображения логического времени на физическое}\label{chap:b}
\addcontentsline{toc}{chapter}{\appendixname~~Модуль отображения логического времени на физическое}

Листинг~\ref{lst:time} содержит функцию для запуска отображения логического времени на физическое. Также определяет в зависимости от знака поданного логического времени какой расчет будет запущен.

\begin{lstlisting}[language=Golang,caption={Публичная функция для запуска извне},label=lst:time]
	func LogicalToPhysTime(physStarts int64, logicalTime int64, config ConverterConfig) (int64, int64) {
		confNum, wasPrinted := checkConfig(config)
	
		if !wasPrinted {
			log.Printf("New config %d: %+v\n", confNum, config)
		} else {
			log.Printf("Found config %d\n", confNum)
		}
		log.Printf("Start time is: %v (%v)", physStarts, time.Unix(physStarts, 0).UTC().Format("02/01/2006 15:04:05"))
		log.Printf("Given logical time is: %v (%v in hours)", logicalTime, logicalTime/secondsInHour)
	
		if logicalTime == 0 {
			first, _ := mapLogicalToPhysTime(physStarts, logicalTime, config, true)
			_, second := mapLogicalToPhysTime(physStarts, logicalTime, config, false)
			return first, second
		} else if logicalTime > 0 {
			return mapLogicalToPhysTime(physStarts, logicalTime, config, false)
		}
		second, first := mapLogicalToPhysTime(physStarts, -logicalTime, config, true)
		return first, second
	}
\end{lstlisting}

Листинг~\ref{lst:map} содержит основной алгоритм отображения. Возвращает два значения: начало и конец интервала, на который отображается логическое время.

\begin{lstlisting}[language=Golang,caption={Алгоритм отображения},label=lst:map]
	func mapLogicalToPhysTime(physStarts int64, logicalTime int64, config ConverterConfig, isReverted bool) (int64, int64) {
		curOffset := int64(0)
		curTime := physStarts
		first := int64(0)
		isLeftFound := false
		log.Println(" Logical time | Physical time")
		log.Println("--------------|----------------------------------")
		for curOffset <= logicalTime {
			curDay := timeToDate(curTime, isReverted)
			intervals := config.getSchedule(curTime, isReverted)
	
			if len(intervals) < 1 {
				curTime = nextDay(curTime, isReverted)
				continue
			}
	
			for _, interval := range intervals {
				log.Printf(" %-13v| %v (%v)", curOffset, curTime, time.Unix(curTime, 0).UTC().Format("02/01/2006 15:04:05"))
				curOffset, curTime = interval.appendIntervalDur(curTime, curOffset, curDay, isReverted)
				if curOffset >= logicalTime {
					curTime = appendTime(curTime, logicalTime-curOffset, isReverted)
					if !isLeftFound {
						first = curTime
						curOffset = logicalTime
						isLeftFound = true
						break
					} else {
						break
					}
				}
			}
		}
		log.Println("Result:")
		log.Println("----------------------------------")
		log.Printf(" %-13v| [%-19v - %-19v]", logicalTime, first, curTime)
		log.Printf(" %-13v| [%-19v - %-19v]", logicalTime, time.Unix(first, 0).UTC().Format("02/01/2006 15:04:05"), time.Unix(curTime, 0).UTC().Format("02/01/2006 15:04:05"))
		return first, curTime
	}
\end{lstlisting}

Листинг~\ref{lst:append} содержит метод реализующий прибавление или вычитание интервала в зависимости от направления расчета.

\begin{lstlisting}[language=Golang,caption={Добавление рабочего интервала},label=lst:append]
	func (interval WorkInterval) appendIntervalDur(physTime int64, logicalTime int64, day int64, isReverted bool) (int64, int64) {
		if isReverted {
			if (interval.Ends + day) >= physTime {
				logicalTime += physTime - day - interval.Starts
			} else {
				logicalTime += interval.Ends - interval.Starts
			}
			physTime = day + interval.Starts
		} else {
			if (interval.Starts + day) <= physTime {
				logicalTime += interval.Ends - physTime + day
			} else {
				logicalTime += interval.Ends - interval.Starts
			}
			physTime = day + interval.Ends
		}
		return logicalTime, physTime
	}
\end{lstlisting}

Листинг~\ref{lst:get} содержит реализацию метода получения смен занятых сотрудников. В зависимости от направления расчета может ``развернуть'' список интервалов.

\begin{lstlisting}[language=Golang,caption={Метод получения расписания смен занятых сотрудников},label=lst:get]
	func (config ConverterConfig) getSchedule(physTime int64, isReverted bool) []WorkInterval {
		currentDay := timeToDate(physTime, isReverted)
	
		scheduleName := config.getTemplateName(currentDay)
		intervals := config.ScheduleForDay.getIntervals(physTime-currentDay, scheduleName, isReverted)
		scheduleName = config.getTemplateName(currentDay - oneDayInSecs)
		prevIntervals := config.ScheduleForDay.getIntervals((physTime-currentDay)+oneDayInSecs, scheduleName, isReverted)
	
		if len(prevIntervals) > 0 {
			for i, interval := range prevIntervals {
				if interval.Starts-oneDayInSecs >= 0 || interval.Ends-oneDayInSecs > 0 {
					prevIntervals[i].Starts = interval.Starts - oneDayInSecs
					prevIntervals[i].Ends = interval.Ends - oneDayInSecs
					intervals = append([]WorkInterval{prevIntervals[i]}, intervals...)
				}
			}
		}
		if isReverted && len(intervals) > 1 {
			last := len(intervals) - 1
			for i := 0; i <= last/2; i++ {
				intervals[i], intervals[last-i] = intervals[last-i], intervals[i]
			}
		}
	
		return intervals
	}
\end{lstlisting}

Листинг~\ref{lst:getI} содержит реализацию извлечения смен из конфигурации по заданному шаблону рабочего дня.

\begin{lstlisting}[language=Golang,caption={Извлечение смен из конфигурации по заданному шаблону},label=lst:getI]
	func (schedules ScheduleForDay) getIntervals(physTime int64, scheduleName DayTemplateName, isReverted bool) []WorkInterval {
		result := make([]WorkInterval, 0)
		intervals := schedules[scheduleName]
	
		for _, interval := range intervals {
			if interval.contains(physTime, isReverted) {
				result = append(result, interval)
			}
		}
		return result
	}
\end{lstlisting}

Листинг~\ref{lst:getT} содержит реализацию получения названия шаблона рабочего дня из даты.

\begin{lstlisting}[language=Golang,caption={Получение названия шаблона рабочего дня из даты},label=lst:getT]
	func (config ConverterConfig) getTemplateName(date int64) DayTemplateName {
		exception, ok := config.ExceptionDates[date]
		if ok {
			return exception
		}
		if exception != "" {
			return exception
		}
		for item := range config.ScheduleTemplate {
			if item == int(time.Unix(date, 0).UTC().Weekday()) {
				return config.ScheduleTemplate[item]
			}
		}
		panic("No schedule defined")
	}
\end{lstlisting}