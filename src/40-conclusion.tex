\section*{Заключение}
\addcontentsline{toc}{chapter}{Заключение}
В рамках данной работы были рассмотрены, разработаны и протестированы компоненты системы планирования производства, а также организованно хранилище данных.

По итогу выполнения работы были достигнуты следующие результаты:
\begin{itemize}
	\item произведен анализ отображения логического на физическое время, синтезирован, реализован и протестирован алгоритм отображения логического времени на физическое;
	\item проанализирована сборочная линия (её функции и ограничения, накладываемые на перемещение продукции), реализована и протестирована модель сборочной линии, впоследствии интегрированная в подсистему имитационного моделирования СПП;
	\item проанализированы данные предприятия, которые необходимо хранить в базе данных, синтезирована и обоснована её структура, проведено доказательство основных положений структуры на основе реляционной теории.
\end{itemize}

\indent Разработанные модули в настоящий момент интегрированы в СПП с соответствующими интеграционными тестами.\\
\indent Структура базы данных была реализована и используется для хранения и извлечения тестовых данных, что используется для тестирования подсистемы имитационного моделирования. 
Также были реализованны введенные в процессе обоснования структуры базы данных операторы и выражения, что упростило работу с версионированными отношениями.

% В данной работе был проведен анализ и разработка компонент системы планирования производства, являющуюся частью интегрируемого программного комплекса интеллектуальной системы управления предприятием, и, впоследствии, были внедрены в данную систему.
\
% \indent В результате был разработан и протестирован алгоритм отображения логического времени на физическое, позволяющий:
% \begin{itemize}
% 	\item производить гибкую настройку своей работы путем изменения конфигурации рабочего календаря;
% 	\item производить как прямой так и обратный расчет;
% 	\item отслеживать корректность работы путем ручного (при помощи логов) и автоматизированного тестирования.
% \end{itemize}
% \to\do[inline]{Написать и исправить}