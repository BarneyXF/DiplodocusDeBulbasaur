\section{Постановка задачи}
Целью данной работы является разработка системных компонент для СПП таких как:
\begin{itemize}
	\item модуль ресурса сборочной линии;
	\item модуль отображения логического времени на физическое;
	\item организация консистентного хранилища данных.
\end{itemize}
% \subsection{Календарь}

% \subsection{Ресурс сборочной линии}
\todo[inline]{Подумать над формулировкой}
Данная компонента является частью (или частной реализацией) модели ресурсов СПП и отражает поведение во времени продуктов на сборочной линии. Основная сложность данной компоненты в необходимости объединения нескольких сборочных линий со схожими параметрами в один ресурс, хранении их состояний, синхронизации между \todo{станциями} и распределении \todo{входного потока продуктов} по линиям. Также, сложностью является сама модель ресурса, которая обеспечивает хорошую масштабируемость, но при этом требует времени на понимание и создание модулей.
% \subsection{База данных}
База данных является основных хранилищем всех постоянных \todo{(не расчетных, как, например, результат работы СПП), так ли это?} данных. При разработке было выделено требование к хранению \todo{'истории'} предыдущих расчетов и их параметров, что влечет к требованию \todo{'консистентности'} \todo{базы данных, БД?}