
\section{Выбор хранилища данных}\label{sec:choose}
\indent Система планирования производства, как и практически любая система, получающая и обрабатывающая данные, нуждается в хранилище данных -- базе данных.
Хранилище данных может быть разделено на две составляющие:
\begin{itemize}
	\item хранилище временных данных;
	\item хранилище постоянных данных.
\end{itemize}

\indent Под временными данными подразумеваются данные, которые получаются в процессе работы системы (например оперативные и объемно-календарные планы) и, при необходимости, могут быть рассчитаны заново, хоть и с некоторыми затратами (время или вычислительные мощности).
В данной работе этот вид данных и хранилище для них не рассматривается.\\
\indent С другой стороны существуют постоянные данные -- информация которая задается, например предприятием, потеря которой в лучшем случае приведет к необходимости заново добавлять их в систему, а в худшем -- приведет к утрате данной информации.
В любом случае потеря постоянных данных ведет к критическим нарушениям в работе системы, что обуславливает необходимость организации консистентного хранилища данных.\\
\indent Консистентность -- требование к данным, получаемым из базы данных, которое заключается в том, что последние должны быть целостны и непротиворечивы.
Под целостностью данных подразумевается соответствие имеющейся в базе данных информации её внутренней логике, структуре и явно заданным правилам.
Любое правило, направленное на ограничение возможного состояния базы данных,называют ограничением целостности.
Помимо целостных, данные должны также быть непротиворечивыми -- это означает, что в базе данных нет логического противоречия, то есть некоторого утверждения и его отрицания.\\
\indent В случае системы планирования производства, в качестве постоянных данных требуется хранить информацию о каждом запуске системы для того, чтобы можно было затем выбрать оптимальный план работы предприятия.
Это ведет к тому, что появляется несколько версий одних и тех же данных и приводит к необходимости организации хранения и извлечения этих версий.
Соответственно, разрабатываемая база данных должна быть доступна только на запись и чтение, что позволит сохранить информацию о предыдущих запусках и добавлять новую.\\
\indent Для организации хранилища, была выбрана реляционная модель базы данных.
Данный выбор обусловлен тем, что данная модель получила широкое распространение, что является сильным аргументом как при внедрении системы (потому что реляционные баз данных организованны на многих предприятиях), так и при разработке, так как это основной тип баз данных при изучении таковых.
Также реляционная модель обладает мощным математическим аппаратом, который основан на теории множеств, что позволяет анализировать, обосновывать и оптимизировать структуру разработанной базы данных и действия над ней.
\todo[inline]{Пример конкретной ТК и эксперимент с разделением что обеспечивает СУБД, а что необходимо обеспечивать операторам}
% \to\do[inline]{почему реляционная - необходимо прочтение}
% \to\do[inline]{ссылки на использованные материалы}


% Со стороны систем управления базами данных (совокупность программных и лингвистических средств общего или специального назначения, обеспечивающих управление созданием и использованием баз данных) свойство консистентности выполняется на уровне транзакций (одним из требований к которой и является консистентность): если одна из команд транзакции не прошла проверку ограничений целостности, то вся транзакция откатывается, то есть база данных возвращается в состояние, в котором была начата транзакция.\\
% \indent Транзакция -- группа последовательных операций с базой данных, которая представляет собой логическую единицу работы с данными.
% Транзакция может либо быть целиком и успешно независимо от идущих параллельно транзакций и соблюдая все ограничения целостности, либо не быть выполненной вообще, что в таком случае не должно оказать на систему никакого влияния.\\
% \indent В соответствии с требованиями описанными выше, можно сделать вывод о необходимости использования реляционной базы данных, одним из преимуществ которой является соответствие требованиям ACID (Atomicity, Consistency, Isolation, Durability -- атомарность, консистентность, которая и интересует в первую очередь, изолированность, долговечность).