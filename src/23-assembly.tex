\section{Модель ресурса сборочной линии}
\indent В процессе создания оперативного плана, для получения адекватной оценки времени выполнения операции (набора операций) СПП необходимо ввести систему ограничений, которая будет отражать как ресурс, участвующий в операции может влиять на её время выполнения.
Это привело к созданию модели ресурсов накладывающей ограничения на выбор операции для расчета ядром имитационного моделирования.\\
\indent Каждый ресурс представляет из себя структуру данных, которая должна реализовывать три метода:
\begin{itemize}
	\item привязка операции к ресурсу;
	\item метод, осуществляющий проверку возможности выполнения данной операции ресурсом;
	\item метод, осуществляющий логику работы и в котором происходит изменение состояния данного ресурса.
\end{itemize}

\indent Привязка осуществляется в начале работы системы, что позволяет ресурсам манипулировать ядром имитационного моделирования разрешая ил запрещая выбирать привязанные к ним операции для расчета, что может повлечь за собой изменение последовательности выполнения операций и, соответственно, расчетного времени выполнения набора операций.\\
\indent Проверка производится во время работы системы и именно здесь происходит отбор операций в соответствии с внутренним состоянием ресурса.\\
\indent Логика осуществляется при выборке операции ядром и для каждой вызывается два раза: чтобы отметить состояние ресурса в начале и в конце расчета операции.\\

% \indent На каждой итерации проверяется наличие привязки ресурсов ко всем доступным для выбора операциям и, при наличии таковых, осуществляется опрос каждого ресурса на то, накладывает ли он какое-либо ограничение на выбор данной операции в текущую итерацию.

\todo[inline]{блок-схема ресурсов}

% \todo[inline]{Подумать над формулировкой}
% Данная компонента является частью (или частной реализацией) модели ресурсов СПП и отражает поведение во времени продуктов на сборочной линии. Основная сложность данной компоненты в необходимости объединения нескольких сборочных линий со схожими параметрами в один ресурс, хранении их состояний, синхронизации между \todo{станциями} и распределении \todo{входного потока продуктов} по линиям. Также, сложностью является сама модель ресурса, которая обеспечивает хорошую масштабируемость, но при этом требует времени на понимание и создание модулей.