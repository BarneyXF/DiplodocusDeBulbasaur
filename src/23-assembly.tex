\section{Модель ресурса сборочной линии}
\indent В процессе создания оперативного плана, для получения адекватной оценки времени выполнения операции или набора операций СПП необходимо ввести систему ограничений, которая будет отражать как ресурс\todo{изменить описание, добавить определение, что я понимаю под ресурсом}, участвующий в операции может влиять на её время выполнения.
Это привело к созданию модели ресурсов накладывающей ограничения на выбор операции для расчета ядром имитационного моделирования.\\
\indent Каждый ресурс представляет из себя структуру данных, которая должна реализовывать три метода:

\begin{itemize}
	\item привязка операции к ресурсу;
	\item метод, осуществляющий проверку возможности выполнения данной операции ресурсом;
	\item метод, осуществляющий логику работы и в котором происходит изменение состояния данного ресурса.
\end{itemize}

\indent Привязка осуществляется в начале работы системы, что позволяет ресурсам манипулировать ядром имитационного моделирования разрешая или запрещая выбирать привязанные к ним операции для расчета, что может повлечь за собой изменение последовательности выполнения операций и, соответственно, расчетного времени выполнения карты технического процесса.\\
\indent Проверка производится во время работы системы и именно здесь происходит отбор операций в соответствии с внутренним состоянием ресурса.\\
\indent Логика осуществляется при выборке операции ядром и для каждой вызывается два раза: чтобы отметить состояние ресурса в начале и в конце расчета операции.\\

\todo[inline]{блок-схема ресурсов}

\indent Одним из ресурсов является ресурс сборочной линии, который описывает несколько однотипных, то есть с одинаковым числом рабочих постов (заготовко-места, оснащенные соответствующим технологическим оборудованием и предназначенными для технического воздействия на заготовку для осуществления фиксированного перечня операций), физических сборочных линий\todo{определение сборочной линии?}.
Объединение нескольких сборочных линий в одну обуславливается упрощением как взаимодействия с ядром имитационного моделирования так и упрощением управления ресурсом потому как даже в худшем случае (когда все линии будут различны по количеству постов) количество ресурсов будет всегда меньше либо равно количеству физических сборочных линии, а также возможностью инкапсуляции реализации распределения заготовок и связанных с ними операций по сборочным линиям внутри ресурса.

\begin{figure}[h]
	\includegraphics[width=\linewidth]{pics/assemblyMain.png}
	\caption{Схема ресурса с вариантами перемещения заготовки внутри}
	\label{fig:assemblyMain}
	% \centering
\end{figure}
\todo{Добавить цифры к линиям, чтобы ссылаться на конкретные}

\indent Главным предназначением данного ресурса является ограничение перемещения продукции внутри ресурса (см. \ref{fig:assemblyMain}), и моделирование работы физического сборочного конвейера.
С одной стороны ограничивается перемещение между сборочными линиями: к какой продукт был привязан, на той он и останется до окончания выполнения всех операций, которые относятся к данному продукту и привязаны к постам данной сборочной линии.
С другой - ограничивается перемещение продукции между рабочими постами: продукт должен двигаться последовательно с поста на пост (см. \ref{fig:assemblyMain}).\\
\indent Одной из ключевых особенностей практически любой сборочной линии является синхронизация передвижения продукции между постами. 
Это означает что такт производства (время,в течение,которого заготовка пребывает на посту) будет равен максимальной временной отметке среди всех постов или другими словами: каждая заготовка сможет сменить пост только после того, как все остальные заготовки будут готовы к смене своих постов.\\
\todo[inline]{схема описания как производится динамическая привязка продукта}
\indent Для реализации необходимого функционала, были введены структуры описывающие очередь продукции, посты линии, саму линию и проекцию привязки операций к линиям\todo{что это, пояснение, ссылка на предыдущий рисунок}.
Каждая из линий ресурса характеризуется временем начала текущего рабочего такта сборочной линии, максимальным временем рабочего такта и набором рабочих постов, каждый из которых описывается состоянием (выполняются работы, простаивает, отсутствует продукция на посту), временной меткой данного поста и продукцией, которая на данный момент находится на нем.
Максимальное время отражает какое время работал пост с начала работы системы.
Так как карта технического процесса позволяет, при возможности, производить параллельные операции над заготовкой, то необходимо знать с какого времени бал начат такт для чего и вводится время начала работы поста.
Время начала текущего рабочего такта показывает время с которого началась работа на текущем посту над текущей заготовкой и для всех постов она равна (из-за синхронизации постов).

Также ресурс имеет информацию о привязке каждой операции каждой единицы продукции к какому-либо рабочему посту (без привязки к конкретной линии) сборочной линии, что дает возможность динамически распределять продукцию между линиями.\\
\indent Во время работы ядра имитационного моделирования на каждой итерации проверяется наличие привязки ресурсов ко всем доступным для выбора операциям и, при наличии таковых, осуществляется опрос каждого ресурса на то, накладывает ли он какое-либо ограничение на выбор данной операции в текущую итерацию.
\todo[inline]{описание синхронизации линии}
\todo[inline]{описание движения линии}

\indent %Основной сложностью реализации данного модуля является предложенная система ресурсов, которая является универсальной и позволяет реализовать логику любого ресурса, при этом усложняя реализацию каждого из них.
% \todo[inline]{Подумать над формулировкой}
% Данная компонента является частью (или частной реализацией) модели ресурсов СПП и отражает поведение во времени продуктов на сборочной линии. Основная сложность данной компоненты в необходимости объединения нескольких сборочных линий со схожими параметрами в один ресурс, хранении их состояний, синхронизации между \todo{постами} и распределении \todo{входного потока продуктов} по линиям. Также, сложностью является сама модель ресурса, которая обеспечивает хорошую масштабируемость, но при этом требует времени на понимание и создание модулей.