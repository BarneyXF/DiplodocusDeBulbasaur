\chapter{Организация консистентного хранилища данных}
\section{Необходимость}\todo{rename}
\indent Система планирования и прогнозирования, как и любая система получающая и обрабатывающая данные, нуждается в хранилище данных - базе данных.
Хранилище данных может быть разделено на две составляющие:
\begin{itemize}
	\item хранилище временных данных;
	\item хранилище постоянных данных.
\end{itemize}

\indent Под временными данными подразумеваются данные, которые получаются в процессе работы системы (например оперативные и объемно-календарные планы) и, при необходимости, могут быть рассчитаны заново, хоть и с некоторыми затратами (время, вычислительные мощности), при условии отсутствия хранилища данных.
В данной работе этот вид данных и хранилище для них не рассматривается.\\
\indent С другой стороны постоянные данные - данные, которые задаются пользователем (в данном случае организацией) и потеря которых в лучшем случае приведет к необходимости заново добавлять их в систему, а в худшем - приведет к утрате этих данных.
В любом случае потеря постоянных данных ведет к нарушениям в работе системы, что ведет к необходимости организации консистентного хранилища данных.\\
\indent Консистентность - требование к данным, получаемым из базы данных, которое заключается в том, что последние должны быть целостны и непротиворечивы.
Под целостностью данных подразумевается соответствие имеющейся в базе данных информации её внутренней логике, структуре и явно заданным правилам.
Любое правило, направленное на ограничение возможного состояния базы данных называют ограничением целостности.\todo{зачем вводить это определение}
Помимо целостных, данные должны также быть непротиворечивыми, что означает, что в базе данных нет логического противоречия, то есть некоторого утверждения и его отрицания.



\todo[inline]{необходимость}
\todo[inline]{математическое обоснование?}
\todo[inline]{схема бд}
% База данных является основных хранилищем всех постоянных \todo{(не расчетных, как, например, результат работы СПП), так ли это?} данных. При разработке было выделено требование к хранению \todo{'истории'} предыдущих расчетов и их параметров, что влечет к требованию \todo{'консистентности'} \todo{базы данных, БД?}