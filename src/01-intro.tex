\section*{Введение}
\addcontentsline{toc}{chapter}{Введение}
\indent \textbf{Актуальность темы исследования.}
В рамках Четвертой промышленной революции или по-другому - Индустрии 4.0 в мире происходит постепенное внедрение киберфизических систем в жизнь человека.
Данная тенденция ведет к растущей потребности в данных системах и их состовляющих: физической (датчики) и программной (вычислительные системы).
Это обуславливает создание интеллектуальных вычислительных систем, которые смогут обеспечить большую точность и скорость планирования деятельности производств, что позволит сократить потери производственных мощностей и ресурсов, а также ускорит актуализацию планов по результатам деятельности предприятия.
Из всего вышесказанного следует необходимость в разработке компонент для данной системы, которые будут решать поставленные задачи, а также, учитывая модульную архитектуру системы, могут быть в будущем использованы в последующих проектах.

% \indent \textbf{Степень теоретической разработанности темы.}
% \todo[inline]{}

\indent \textbf{Цель работы.} 
Целью данной работы является разработка и тестирование компонент для системы планирования производства, для чего были поставлены следующие задачи:
\begin{itemize}
	\item разработка модели ресурса сборочной линии;
	\item разработка модуля отображения логического времени на физическое;
	\item организация консистентного хранилища данных;
	\item тестирование полученных модулей и верификация структуры базы данных.
\end{itemize}

\indent \textbf{Теоретическая и практическая значимость.}
Результаты данного исследования могут быть применены:
\begin{itemize}
	\item при разработке программных комплексов управления предприятиями;
	\item при разработке имитационных моделей производства;
	\item при разработке структуры консистентного хранилища данных.
\end{itemize}

\indent \textbf{Методы исследования.}
Для достижения поставленных задач используются следующие методы:
\begin{itemize}
	\item анализ логического и физического времён, их различий и синтез соответствующего алгоритма отображения логического времени на физическое;
	\item анализ сборочной линии, синтез модели сборочной линии и моделирование в подсистеме имитационного моделирования;
	\item анализ данных, которые необходимо хранить в базе данных, синтез структуры базы данных, формализация и обоснование предложенной структуры.
\end{itemize}

% \indent \textbf{Положения выносимые на защиту.}
% \todo[inline]{}
\indent \textbf{Апробация результатов исследования.}
Результаты исследования используются в основной составляющей интегрируемого программного комплекса интеллектуальной системы управления предприятием (ИСУП) – системе планирования и прогнозирования\todo{или производства?} (СПП). 
Результаты исследований по отображению логического времени на физическое были представлены в статье для Конгресса молодых ученых (2019) ``Задача отображения временных промежутков на рабочий календарь''
\todo[inline]{fix subsections}