\section*{Введение}
\addcontentsline{toc}{chapter}{Введение}
\indent \textbf{Актуальность темы исследования.}
В рамках Четвертой промышленной революции или по-другому -- Индустрии 4.0 в мире происходит постепенное внедрение киберфизических систем в жизнь человека.
Данная тенденция ведет к растущей потребности в данных системах и их состовляющих: физической (датчики) и программной (вычислительные системы).
Это обуславливает создание интеллектуальных вычислительных систем, которые смогут обеспечить большую точность и скорость планирования деятельности производств, что позволит сократить потери производственных мощностей и ресурсов, а также ускорит актуализацию планов по результатам деятельности предприятия.
Из всего вышесказанного следует необходимость в разработке компонент для данной интеллектуальной системы, которые будут решать поставленные задачи, такие как моделирование производственных ресурсов, преобразование выходных данных, хранение данных и другие.\\
\indent \textbf{Цель работы.}
Целью данной работы является разработка и тестирование компонент системы планирования производства.
Были поставлены следующие задачи:
\begin{itemize}
	\item разработка модели ресурса сборочной линии;
	\item разработка модуля отображения логического времени на физическое;
	\item организация консистентного хранилища данных;
	\item тестирование полученных модулей и верификация принципов формирования структуры базы данных.
\end{itemize}

\indent \textbf{Теоретическая и практическая значимость.}
Результаты данного исследования могут быть применены:
\begin{itemize}
	\item при разработке программных комплексов управления предприятиями;
	\item при разработке имитационных моделей производства;
	\item при разработке структуры консистентного хранилища данных.
\end{itemize}

\indent \textbf{Методы исследования.}
Для достижения поставленных задач используются следующие методы:
\begin{itemize}
	\item реляционная алгебра;
	\item теория множеств;
	\item структурное программирование;
	\item тестирование программного обеспечения.
	% \item анализ сборочной линии, синтез модели сборочной линии и моделирование в подсистеме имитационного моделирования;
	% \item анализ логического и физического времён, их различий и синтез соответствующего алгоритма отображения логического времени на физическое;
	% \item анализ данных, которые необходимо хранить в базе данных, синтез структуры базы данных, формализация и обоснование предложенной структуры.
\end{itemize}
% \to\do{Раздел о том, какие методы Вы использовали: реляц. алгебра, теория множеств, структурное программирование...}

\indent \textbf{Апробация результатов исследования.}
Результаты исследования используются в интегрируемом программном комплексе интеллектуальная система управления предприятием (ИСУП) – системе планирования производства (СПП).%\to\do{или прогнозирования?}
Результаты исследований по отображению логического времени на физическое были представлены в статье для Конгресса молодых ученых (2019) ``Задача отображения временных промежутков на рабочий календарь''