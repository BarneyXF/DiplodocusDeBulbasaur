\section{Архитектура системы планирования производства}
\indent Система планирования производства представляет из себя набор программных модулей, взаимодействующих согласно схеме (рисунок \ref{fig:archSPP}).

\begin{figure}[ht]
	\centering
	\includegraphics[width=\linewidth]{pics/archSPP.png}
	\caption{Схема системы планирования производства \cite{niorkpz}}
	\label{fig:archSPP}
\end{figure}

\indent Так как для выбора оптимального плана сотруднику требуется несколько планов, непосредственно из которых нужно будет оптимальный\todo{???}, система производит запуск множества параллельных расчетов, каждый из которых отличается конфигурацией смен либо количеством доступных ресурсов.
За запуск и синхронизацию отвечает ``Модуль управления СПП'' (рисунок \ref{fig:archSPP}), который создает очередь расчетов на запуск.
Затем пул потоков (автоматическое средство для задач, которые требуют временных запусков потоков) извлекает их в порядке очереди и запускает работу подсистемы имитационного моделирования в отдельном потоке (``Алгоритмы и модели без состояний'', рисунок \ref{fig:archSPP}).
После окончания работы полученный результат передается обратно в модуль управления для возврата пользователю посредством RESTful API (веб-служба, построенная с учетом архитектурного стиля REST~--~ стиль взаимодействия компонентов распределённого приложения в сети).
Пользователь, зная с какими параметрами запускался полученный расчет, может поменять конфигурацию и отправить его на повторное вычисление.
Конфигурация измененного расчета будет сохранена в базе данных предприятия и приведет к запуску всего цикла с начала.