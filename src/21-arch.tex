\section{Архитектура системы планирования производства}
\indent Система планирования производства представляет из себя набор программных модулей, как представлено на рисунке \ref{fig:archSPP}.

\begin{figure}[ht]
	\centering
	\includegraphics[width=\linewidth]{pics/archSPP.png}
	\caption{Схема системы планирования производства \cite{niorkpz}}
	\label{fig:archSPP}
\end{figure}

\indent Так как для выбора оптимального плана человеку требуется несколько планов, собственно из которых и нужно будет выбрать нужный, система производит запуск множества параллельных расчетов, каждый из которых отличается конфигурацией смен либо количества доступных ресурсов.
За запуск и синхронизацию отвечает "Модуль управления СПП" (\ref{fig:archSPP}), который создает очередь расчетов на запуск.
Затем пул потоков извлекает их в порядке очереди и начинает работу ядра имитационного моделирования ("Алгоритмы и модели без состояний", рисунок \ref{fig:archSPP}).
После окончания работы полученный результат передается обратно в модуль управления для возврата пользователю посредством RESTful API(REST~-~архитектурный стиль взаимодействия компонентов распределённого приложения в сети).
Пользователь, зная с какими параметрами запускался полученный расчет, может поменять конфигурацию и отправить на повторное вычисление, что будет сохранено в базе данных предприятия и запустит весь цикл заново.