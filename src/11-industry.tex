\section{Индустрия 4.0}
Индустрия 4.0 - прогнозируемое массовое внедрение киберфизических систем в промышленность и повседневный быт человека.
Под киберфизической системой (CPS - cyber-physical system) подразумевается взаимодействие цифровых, аналоговых, физических и человеческих компонентов, разработанных для функционирования посредством интегрированной физики и логики или другими словами: киберфизические системы - это интеллектуальные системы, которые состоят из тесно взаимосвязанных сетей физических и вычислительных компонентов.\cite{nist}\\
\indent Появилось понятие индустрии 4.0 во время ганноверской выставки 2011 года как обозначение стратегического плана развития и поддержания конкурентоспособности немецкой экономики, предусматривающий совершение прорыва в области информационных технологий для промышленности.
Также считается, что данное направление знаменует собой четвертую промышленную революцию.\cite{industry}\\
\indent Если рассматривать производство, на что и нацелена индустрия 4.0, то, используя определение данное выше, процесс внедрения киберфизической системы разделяется на внедрение физической части (например датчики, собирающие и передающие данные посредством различных сетей) и вычислительную - систему планирования производства.\\
\indent Система планирования производства (СПП) обеспечивает расчет объемно-календарного и оперативного планов, автоматизированный подбор поставщиков, автоматизированный перерасчет планов по фактическим результатам деятельности, направленный на минимизацию временных и финансовых потерь.\\
\indent Объемно-календарный план - задание для каждой производственной площадки на заданный интервал времени, представляющее собой план-график, на котором каждому интервалу соответствует номенклатура и объем подлежащих к производству изделий.\cite{niokr}\\
\indent Оперативный план - план, согласно которому выполняется привязка каждой операции для каждой единицы продукции к временным интервалам, конкретному работнику и конкретным производственным средствам.\cite{niokr}\\
\indent С небольшими оговорками можно сказать, что оперативный план является более подробной версией объемно-календарного плана, которая, в силу ограничений со стороны вычислительной сложности расчета, составляется на срок в порядок меньший, чем для объемно-календарного плана.
\todo[inline]{схема и сравнение планов}
