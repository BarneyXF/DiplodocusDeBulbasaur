\section{Обзор существующих систем планирования производства}
\indent Сегодня, чтобы сохранять конкурентоспособность, предприятиям необходимо развивать и внедрять системы управления и планирования производственных процессов. 
Актуальным направлением, ориентированным на решение этой задачи, является разработка СПП, которая должна обеспечивать решение следующих задач: исполнения планов производства и целевых показателей, оптимизации оперативной деятельности, снижения затрат и повышения эффективности производства.\\
\indent Производственное планирование — это систематический, структурированный, направленный на достижение поставленной задачи процесс планирования промышленного предприятия, состоящий из отдельных этапов, осуществляемый с помощью специальных методов и инструментов, начиная с замысла и заканчивая запуском производства \cite{mulBook}.
Производственное планирование может также включать в себя корректирующие мероприятия в процессе эксплуатации. Планирование промышленного предприятия может основываться на различных целях и задачах, охватывать самые разные производственные ситуации.\\
\indent В процессе планирования используется ряд инструментов и программных компонент, которые поддерживают реализацию метода планирования \cite{lodonBook}.
Они применяются пользователями на персональных компьютерах в различных фазах планирования промышленного предприятия.
К основным инструментам планирования относятся: APS/SCM; САР; АСУП. 
Ниже даны подробные описания данных типов инструментов \cite{oliriBook}.\\
\indent APS/SCM (системы синхронного планирования, системы управления логической цепочкой) поддерживают расчеты, эксплуатацию и оптимизацию логистических цепочек. 
Особые свойства логистической цепочки получаются благодаря взаимодействию участников. 
Важная роль отводится структуре логистических цепочек, соответствующих рынку, а также координации и интеграции всех индивидуальных действий.\\
\indent САР (система автоматизированного регулирования) характеризует область компьютеризованного планирования работы. 
При этом используется система электронной обработки данных для создания рабочего графика, выбора эксплуатационных средств, создания указаний по изготовлению и монтажу, а также программирования для станков с ЧПУ.\\
\indent АСУП (САМ — автоматическая система управления производством) включает в себя компьютерное техническое управление и контроль над производственными линиями и эксплуатационными средствами при проведении производства, т.е. прямое управление обрабатывающими и перерабатывающими машинами, устройствами манипуляции, транспортировки, перегрузки и хранения, а кратко — техническое управление всеми устройствами потоковых систем \cite{gibBook}.\\
\indent Наиболее распространенными программными продуктами, предназначенными для планирования производственных процессов являются «1С:Предприятие» и «SAP R/3»: первый является наиболее распространенным на российском рынке, второй, в свою очередь, широко используется за рубежом. 
На примере этих, зарекомендовавших себя с положительной стороны продуктов, проведем анализ системных компонент и сравним их место в архитектуре систем.

\subsection{«1С: Предприятие 8.0»}

\indent Комплекс программ «1С: Предприятие 8.0» состоит из технологической платформы и прикладных компонентов, которые создаются на её основе и предназначены для автоматизации деятельности предприятий. 
Технологическая платформа не является готовым программным продуктом, предназначенным для внедрения на предприятие, вместо нее обычно применяют несколько компонентов, разработанных на её базе.
Данное решение делает возможным автоматизировать различные виды деятельности, применяя единую основную технологическую платформу.\\
\indent Система «1С: Предприятие 8.0» использует следующие основные компоненты:

\begin{itemize}
	\item «Управление торговлей»;
	\item «Управление персоналом»;
	\item «Управление производственным предприятием»;
	\item «Управление складом»;
	\item «Управленческий учет и расчет себестоимости».
\end{itemize}

\indent Наиболее интересными для рассмотрения являются компоненты «Управление персоналом» и «Управление производственным предприятием», поскольку их реализация является ключевой с точки зрения планирования деятельности предприятия и не имеет сегодня строгой математической формализации.\\
\indent Компонент «1С: Предприятие 8.0. Управление персоналом» позволяет эффективно управлять кадровыми процессами в следующих областях: планирование потребностей в персонале; обеспечение организации новыми кадрами; эффективное планирование занятости персонала; кадровый учет и анализ персонала; управление персоналом.\\
\indent Компонент «Управление производственным предприятием» предназначен для автоматизации процессов управления и учета на производственном предприятии.
Он позволяет создать единую информационную систему для управления различными сторонами деятельности предприятия.\\
\indent В платформе «1С: Предприятие 8.0» заложен ряд подходов, которые формируют основную концепцию разработки типовых компонентов. 
Эти подходы предназначены для максимального сближения технологических возможностей с бизнес-процессами разработки и интеграции прикладных решений.
Важными моментами, которые следует отметить, являются: изоляция разработчика от технологических деталей, алгоритмическое программирование конкретной бизнес-логики приложения, использование собственной модели базы данных и гибкость прикладных решений без их доработки.\\
\indent Механизм обмена данными, используемый в технологической платформе «1С: Предприятие 8.0», позволяет создавать территориально распределенные информационные системы на основе баз данных «1С: Предприятия 8.0», и использовать другие информационных систем, не относящиеся к «1С: Предприятии 8.0».
Например, можно организовать работу главного офиса, филиалов и складов предприятия в одной базе данных, или обеспечить взаимодействие базы данных «1С: Предприятия 8.0» с существующей базой данных «Oracle».\\
\indent Технологическая платформа «1С: Предприятие 8.0» предоставляет средства разработки, с помощью которых создаются новые или модифицируют существующие прикладные решения.
Этот инструмент разработки называется «конфигуратор».
Благодаря тому, что он поставляется со стандартным пакетом «1С: Предприятия 8.0», то пользователь может свободно разработать или модифицировать прикладное решение (адаптировать его под себя), возможно, с привлечением сторонних специалистов.\\
\indent Среди преимуществ данной системы можно выделить:

\begin{itemize}
	\item открытость системы;
	\item регулярные программные обновления;
	\item широкие функциональные возможности системы.
\end{itemize}


\indent Однако необходимо отметить, что подходы «1С: Предприятия» ориентированы на решение проблем автоматизации бухгалтерского и организационного управления предприятием.
Использование проблемно-ориентированных объектов позволяет разработчику решать задачи складского, бухгалтерского, управленческого учета, расчетам заработной платы, анализа данных и управлению бизнес-процессами.
Однако области экономического и бухгалтерское учета характеризуются высокой степенью математического формализации и их реализация происходит с относительно малыми трудозатратами, тогда как компоненты планирования и производственного расписания сегодня являются актуальными направлением для исследований и прикладной разработки.

\subsection{«SAP R/3»}

\indent Система «SAP R/3» предоставляет собой набор разноплановых инструментов, направленных на повышение эффективности производственного процесса, увеличение экономической стабильности, автоматизацию процессов планирования.
Она дает возможность интегрировать инновационные подходы централизованного планирования и управления, и повысить качество управления на разных организационных уровнях предприятий.\\
\indent «SAP R/3» использует модульную архитектуру, где каждый отдельный модуль предназначен, для решения специализированной задачи процесса предприятия, взаимодействие между ними происходит в режиме реального времени. 
Наибольший интерес для рассмотрения представляют компоненты, не имеющие прямого отношение к бухгалтерской и экономической деятельности предприятия -- модуль PP; модуль HR; модуль BC.\\
\indent Модуль PP (планирование производства) дает возможность организовать управление и планирование производства предприятия.
Он реализует следующие функции: формирование спецификаций, создание технологических карт, управление производственными площадками, планирование сбыта, планирование потребности в материалах, управление производственными заказами, планирование затрат на изготовление изделие, учет затрат производственных процессов, планирование производственной деятельности, управление серийным производством, планирование автоматизированного производства.\\
\indent Модуль HR (управление персоналом) решает задачи планирования и управления работой персонала. 
Ключевые элементы: администрирование персонала, расчет данных для вычисления заработной платы, сбор и анализ данных о рабочем времени, учет командировочных расходов, создание информационной модели внутренней структуры компании.\\
\indent Модуль BC (базовый модуль) предназначен для интеграции в систему «SAP R/3» всех отдельных прикладных модулей и обеспечивает независимость от аппаратной платформы. 
Модуль BC позволяет организовать работу c многоуровневой распределенной архитектуре клиент-сервер. 
Система «SAP R/3» работает на серверах UNIX, AS/400, Windows NT, S/390 и с различными СУБД (Informix, Oracle, Microsoft SQL Server, DB2). \cite{mazBook}.\\
\indent На данный момент система «SAP R/3» является наиболее распространенной системой управления предприятием. 
Благодаря тому, что она является модульной системой, ее можно настроить в соответствии с конкретными потребностями отдельного предприятия. 
Степень технического уровня системы определяется возможностью ее перенастройки без необходимости переписывать программный код. 
Эта опция «SAP R/3» также позволяет занимать ведущее место в мире в системе управления.\\
\indent С помощью инструментов управления, включенных в систему «SAP R/3», можно реализовывать задачи мониторинга и анализа, без дополнительного программирования, для чего система предлагает следующие способы:

\begin{itemize}
	\item мониторинг БД; 
	\item мониторинг операционной системы сервера; 
	\item мониторинг коммуникаций;
	\item мониторинг и управление сервером приложений:
	\begin{itemize}
		\item формирование и запуск новой конфигурация ядра R/3; 
		\item снятие и редактирование текущей конфигурации ядра R/3; 
		\item формирование временного графика в зависимости от нагрузки (например, в ночное время можно увеличивать количество процессов, отвечающих за фоновые задания); 
		\item управление системой архивирования; 
		\item управление текущими пользователями, процессами. 
	\end{itemize}
\end{itemize}

\indent Многоуровневая клиент-серверная архитектура позволяет разделять задачи управления данными, ориентированные на нужды пользователя.
В версии 3.0 системы «SAP R/3» SAP AG были расширены возможности решения для организации взаимодействия с другими приложениями и распределения операций «SAP R/3» в масштабируемой компьютерной структуре. 
Технология внедрения «SAP R/3» основана на многоуровневой архитектуре с использованием программного обеспечения среднего уровня. 
Между тем, промежуточное ПО отделяет пользовательские приложения от аппаратного и программного обеспечения, с другой стороны, решает проблему взаимодействия программных приложений и аппаратного обеспечения. 
В «SAP R/3» SAP Basis функционирует как промежуточное программное обеспечение \cite{mazBook}.\\
\indent Основные данные — это информация, которая хранится в базе данных, достаточно долгий промежуток времени. 
К ним относятся такие данные как: информация о кредиторах, поставщиках, материалах и счетах. 
Основные данные создаются централизованно и доступны для всех приложений. 
Например, они включают данные клиента, которые используются в заявках, поставках, для выставления счетов и платежей и так далее. 
Основные данные клиента могут быть присвоены следующим организационным единицам: балансовая единица, сбытовая организация, канал сбыта, сектор.\\
\indent Основная запись материала является центральным объектом данных системы «SAP R/3». 
Она включает в себя: сырье; оборудование; расходные материалы; полуфабрикаты; продукты; вспомогательное производственное оборудование и инструменты. 
Она является главным источником данных предприятия и используется всеми компонентами логистической системы SAP. 
Благодаря объединению всех материалов в единый объект базы данных устраняются проблемы избыточности данных. 
Сохранённые данные могут использоваться во всех областях, такими как закупки, контроль запасов, планирование потребностей в материалах, проверка счетов.\\
\indent Данные, хранящиеся в основной записи материала, необходимы логистическому модулю системы для решения следующих задач:
\begin{itemize}
	\item обработки запасов на поставку;
	\item обновления движения материалов и инвентаризационной обработки;
	\item проводки счетов;
	\item обработки клиентских заказов;
	\item планирования потребностей и календарного планирования.
\end{itemize}
Структурная логика поставщика и клиента также применяется к основной записи материала. 
При оформлении заказа для клиентов необходимо учитывать: согласование о перевозке, условия доставки, оплаты и т.д. 
Данные, необходимые для таких операций, дублируются из основной записи делового партнера, чтобы исключить необходимость повторного ввода информации о каждой транзакции. 
В основной записи материала могут одновременно храниться данные, обработанные во время ввода заказа, например, цена за единицу цены товара, запасы на другом складе и т.д. 
Этот принцип полезен для обработки данных в каждой основной записи, связанной с выполнением операции.\\
\indent Для каждой транзакции необходимо присваивать соответствующую организационную единицу. 
Присвоение структуры предприятия генерируется в дополнение к данным, доступным по данным клиента и по данным материала. 
Поэтому документ, созданный при помощи транзакции, содержит все основные данные из организационных единиц \cite{gehBook}.\\
\indent Среди достоинств данной системы можно выделить:

\begin{itemize}
	\item прозрачность деятельность предприятия;
	\item повышение оборотов товарно-материальных запасов;
	\item сокращение персонала управления;
	\item единые стандарты управления.
\end{itemize}

\indent К недостаткам относятся:

\begin{itemize}
	\item требуется высокий уровень подготовки персонала;
	\item сложность интеграции;
	\item большие финансовые вложения.
\end{itemize}
