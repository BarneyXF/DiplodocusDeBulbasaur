% \chapter{Глава 3}
% \section{Эксперименты (тесты частей)}
% \section{Результаты}
% \subsection{Календарь}
% В результате был разработан модуль, автоматизирующий расчет временных линий, что позволяет произвести оценку заданного расписания с целью максимизации эффективности (с точки зрения стоимости хранения или сроков) распределения производственных мощностей, задействованных при выполнении заказа. Так же имея конкретные даты, производитель, с некоторой степенью точности может говорить планируемых сроках окончания выполнения заказа.
% \todo[inline, color=red]{10 страниц}
% \subsection{Ресурс сборочной линии}
% \subsection{База данных}

\chapter{Организация консистемного хранилища данных}
\section*{Заключение}
\addcontentsline{toc}{chapter}{Заключение}
\section*{Список литературы}
\addcontentsline{toc}{chapter}{Список литературы}