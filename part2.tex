\chapter{Система планирования и прогнозирования}
\section{Архитектура СПП}
\section{Подсистема имитационного моделирования}
\section{Модель ресурса сборочной линии}
\section{Модуль отображения логического времени}

нужна схема и немного воды
% \section{Календарь}
% \todo[inline, color=yellow]{Схема 'функций'?}
% \subsection{Модель}
Для отображения логического времени на физическое был предложен итеративный процесс, который осуществляет 'переход' к необходимому времени путем последовательного перебора.
\newline
Как было сказано ранее, из-за того, что рабочее время является дискретным, то мы не имеем возможности просуммировать начальную дату и значение поданного логического времени. Это ведет к тому, что необходимо синхронизировать логическое и физическое время, и в данной компоненте это достигается периодическим отображением конкретного логического времени на физическое с использованием информации о переносах дней, сменном графике задействованного персонала и так далее (будет обозначаться как 'конфигурация модуля').\todo{схема с осями}
\newline
Это подразумевает под собой наличие двух 'осей': оси логического времени, которая начинается с нуля и единица которой соответствует одной секунде (необходимости в более точном отображении нет) и оси физического времени, на которой может быть отложено любая дата физического времени, отсчет которой начинается 1 января 1970 года 00:00:00. Особенностью оси физического времени является наличие на ней 'выколотых' промежутков времени, в которые работа не ведется и операции не выполняются и следовательно об этих промежутках системе необходимо знать, и они передаются системе в виде структуры данных(расписать!!!).
\newline
% \subsection{Входные и выходные данные}
Входными данными для модуля являются:
\begin{itemize}
	\item дата с которой необходимо начинать отсчет;
	\item логическое время, которого необходимо достигнуть;
	\item конфигурация модуля.
\end{itemize}
Дата является точкой на физической оси, на которую будет отображаться нуль логической. Представляет собой количество секунд, прошедшее с начала 'Эпохи Unix' - 1 января 1970 года 00:00:00.
Логическое время~-~количество секунд, которое должно быть отложено на логической оси. В силу дискретности физической оси, каждой логической точке сопоставляется отрезок на физической оси, сопоставляется пара чисел - границ данного отрезка.
Конфигурация модуля~-~вспомогательные данные используемые для определения модулем какие промежутки необходимо пропускать в процессе работы. Состоит из данных о сменном графике занятого персонала (шаблонные интервалы рабочего времени), шаблонном расписании на неделю (например, суббота, воскресенье - выходные, пятница - 'короткий' день, остальные - стандартные рабочие дни) и набор информации о датах, которые являются днями-исключениями и соответствующими шаблонами работы в эти дни.
\newline
% \subsection{Результат работы}
Выходными данными данного модуля является пара чисел, характеризующие начало и конец отрезка которые отображаются на логическую ось в точке, значение которой равно входному логическому времени.
\todo[inline]{Схема с осями и парой чисел}

% \subsection{Реализация}
После получения параметров совершается проверка последних на корректность и непротиворечивость (например, если два дня имеют пересечения временных промежутков то они противоречивы, ведь ресурс не может работать одновременно в двух сменах) как в рамках смен одного так соседних дней.

Далее производится определение режима работы: прямой, обратный расчет или проверка времени(нужно более емкое понятие).
Прямой расчет - задается дата начала отсчета, логическое время и расчет ведется до нахождения даты окончания работ.(картинка)
Обратный расчет - задается дата окончания отсчета ('дедлайн'), логическое время и расчет ведется до нахождения начала работ. (картинка)
\todo[inline]{можно дать пояснение зачем это нужно}
Проверка времени - задается дата и логическое время равное нулю, что запускает оба предыдущих расчета пока не будет найдено первое ненулевое время в обоих направлениях от даты расчета. (нужна картинка)
\todo[inline]{можно дать пояснение зачем это нужно}

Выбрав режим работы сбрасывается счетчик текущего логического времени до нуля и счетчик текущего физического времени до стартовой даты. Затем итеративно, пока текущее логическое время не превысит необходимое производится поиск следующей даты. Алгоритмически, поиск даты работает следующим образом:
\begin{enumerate}
	\item[1)] определяются интервалы рабочих смен относящихся к текущему дню:
	      \begin{itemize}
		      \item при отсутствии таковых, к текущей дате прибавляется один день и затем возврат к п.1.
	      \end{itemize}
	\item[2)] отсортированные в порядке возрастания, интервалы перебираются и последовательно их длительности прибавляются к логическому и физическому времени:
	      \begin{itemize}
		      \item при превышении текущим логическим временем необходимого, переход к п.3;
		      \item если все интервалы были использованы, но необходимое логическое время не превышено - переход к п.1;
	      \end{itemize}
	\item[3)] вычитается из физического времени разность текущего и необходимого логического времён, при этом сохраняя данное значение как левую (правую при обратном расчете) и продолжается расчет для выявления правой границы промежутка(пояснение)
\end{enumerate}
(картинка)\\
Определение интервалов рабочего времени происходит взятием даты (важно!!!) из текущего физического времени, после чего начинается определение является ли данная дата одной из перенесенных после чего есть два варианта развития ситуации:
% \begin{enumerate}
% \end{enumerate}


\todo[inline]{последовательным итеративным суммированием}


\todo[inline]{Так как расчет расписания — это итеративный процесс, то в рамках разработки было выделено понятие временной линии (прямой) – это «линия» на которой для каждой точки, которая является абстрактной величиной времени выполнения операции, сопоставляются две даты соответствующие данной абстрактной величине времени с учетом расписания. Первая дата является концом данной операции, вторая – началом следующей. Данное разделение было использовано, потому как все время что между ними также относится к данной точке, а значит каждой точке, из-за непрерывности времени, соответствует бесчисленное множество точек на временной прямой, что может быть лишь ограничено двумя границами – временем начала и конца данного отрезка.}
% \section{Сборочная линия и модель ресурсов}
Немного текста, объединение с архитектурой по количеству страниц?
% \section{База данных}
Можно много наговорить
\todo[inline, color=red]{15 страниц}