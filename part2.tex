\chapter{Глава 2}
\section{Архитектура СПП}
нужна схема и немного воды
\section{Календарь}
\todo[inline, color=yellow]{Схема 'функций'?}
\subsection{Идея}
Для отображения логического времени на физическое был предложен итеративный процесс(\todo{что это}), который осуществляет 'поиск' необходимой даты путем последовательного перебора дат.
\newline
Как было сказано выше, из-за того, что рабочее время не является непрерывным и имеет промежутки обеденных перерывов, выходных, переносы дней и так далее, то мы не имеем возможности просуммировать начальную дату и значение поданного логического времени и это ведет к тому, что необходимо синхронизировать логическое и физическое время, что в данной компоненте достигается периодическим (например, раз в сутки (86400 секунд)) отображением конкретного логического времени на физическое с использованием информации о переносах дней, сменном графике задействованного персонала и так далее ('данные'\todo{как это обозначить}).
\newline
\todo{схема с осями}
Это подразумевает под собой наличие двух 'осей': оси логического времени, которая начинается с нуля и единица которой соответствует одной секунде (необходимости в более точном отображении пока нет, но при возникновении последней переход не потребует больших трудозатрат \todo{кривая формулировка}) и оси физического времени, на которой может быть отложено любая дата физического времени, с отсчетом, начинающимся 1 января 1970 года (\todo{эпоха unix})
Ось логического времени непрерывна (нет 'выколотых' точек, которые не учитываются) в отличие от физического, в котором некоторые точки выколоты и которые система не должна учитывать.
\newline
\subsection{Реализация}
На вход компоненте подается дата и время начала отсчета (в секундах), необходимое для отображения логическое время и 'данные'. 
\todo[inline]{последовательным итеративным суммированием}



\todo[inline]{Так как расчет расписания — это итеративный процесс, то в рамках разработки было выделено понятие временной линии (прямой) – это «линия» на которой для каждой точки, которая является абстрактной величиной времени выполнения операции, сопоставляются две даты соответствующие данной абстрактной величине времени с учетом расписания. Первая дата является концом данной операции, вторая – началом следующей. Данное разделение было использовано, потому как все время что между ними также относится к данной точке, а значит каждой точке, из-за непрерывности времени, соответствует бесчисленное множество точек на временной прямой, что может быть лишь ограничено двумя границами – временем начала и конца данного отрезка.}
\section{Сборочная линия и модель ресурсов}
Немного текста, объединение с архитектурой по количеству страниц?
\section{База данных}
Можно много наговорить
\todo[inline, color=red]{15 страниц}