\chapter{Глава 2}
\section{Архитектура СПП}
нужна схема и немного воды
\section{Календарь}


Так как расчет расписания — это итеративный процесс, то в рамках разработки было выделено понятие временной линии (прямой) – это «линия» на которой для каждой точки, которая является абстрактной величиной времени выполнения операции, сопоставляются две даты соответствующие данной абстрактной величине времени с учетом расписания. Первая дата является концом данной операции, вторая – началом следующей. Данное разделение было использовано, потому как все время что между ними также относится к данной точке, а значит каждой точке, из-за непрерывности времени, соответствует бесчисленное множество точек на временной прямой, что может быть лишь ограничено двумя границами – временем начала и конца данного отрезка.
\section{Сборочная линия и модель ресурсов}
\section{База данных}
\todo[inline, color=red]{15 страниц}